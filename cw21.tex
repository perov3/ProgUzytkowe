\documentclass[12pt, letterpaper, titlepage]{article}
\usepackage[left=3.5cm, right=2.5cm, top=2.5cm, bottom=2.5cm]{geometry}
\usepackage[MeX]{polski}
\usepackage[utf8]{inputenc}
\usepackage{graphicx}
\usepackage{enumerate}
\usepackage{amsmath} %pakiet matematyczny
\usepackage{amssymb} %pakiet dodatkowych symboli
\title{Pierwszy dokument LaTeX}
\author{Patryk Śmiłek}
\date{Październik 2022}
\begin{document}
\maketitle




\begin{table}[h]
\centering\caption{Przykładowy system decyzyjny}
\centering
\begin{tabular}{c|c c c}
\hline
\hline
Pacjent & Ból brzucha & Temperatura ciała & Operacja\\
\hline
u1 & mocny & wysoka & tak\\
u2 & średni & wysoka & tak\\
u3 & mocny & średnia & tak\\
u4 & mocny & niska & tak\\
u5 & średni & średnia & tak\\
u6 & średnia & średnia & nie \\
u7 & mały & wysoka & nie \\
u8 & mały & niska & nie\\
u9 & mocny& niska & nie\\
u10 & mały& średnia & nie\\
\hline
\hline
\end{tabular}
\end{table}

\begin{table}[h]
\centering\caption{ tabela przedstawiające działania bramek logicznych:AND, NAND, OR, NOR,
XOR}
\centering
\begin{tabular}{|c c|c|c|c|c|c|}
\hline
A & B & AND & NAND & OR & NOR & XOR\\
\hline
0 & 0 & 0 & 1 & 0 & 1 & 0\\
0 & 1 & 0 & 1 & 1 & 0 & 1\\
1 & 0 & 0 & 1 & 1 & 0 & 1\\
1 & 1 & 1 & 0 & 1 & 0 & 0\\
\hline
\end{tabular}
\end{table}

\begin{table}[h]
\centering\caption{ tabela przedstawiające działanie bramki logicznej: NOT}
\centering
\begin{tabular}{|c|c| }
\hline
A & NOT \\
0 & 1\\
1 & 0\\
\hline
\end{tabular}
\end{table}


\end{document}