\documentclass[12pt, letterpaper, titlepage]{article}
\usepackage[left=2.5cm, right=2.5cm, top=2.5cm, bottom=2.5cm]{geometry}
\usepackage[MeX]{polski}
\usepackage[utf8]{inputenc}
\usepackage{graphicx}
\usepackage{enumerate}
\usepackage{amsmath} %pakiet matematyczny
\usepackage{amssymb} %pakiet dodatkowych symboli
\title{Kolokwium pierwsze}
\author{Patryk Śmiłek}
\date{06.12.2022}
\begin{document}
\maketitle
\section{Szyfr AES}
Standard  AES  (Advanced  Encryption  Standard)  opublikowany  został  przez  NIST  w  2001
roku. AES  jest  szyfrem  blokowym,  zaprojektowanym  w  celu  zastąpienia  szyfru  DES  jako
przyjęty  standard  w  różnorodnych  zastosowaniach.  W  porównaniu  z  szyframi  z  kluczami
publicznymi - jak RSA, AES (jak zresztą większość szyfrów symetrycznych) jest nieporów-
nanie bardziej skomplikowany i nie da się wyjaśnić tak prosto jak wiele innych algorytmów
kryptograficznych. Względna prostota tej wersji umożliwia ręczne wykonywanie szyfrowania
i deszyfracji, dzięki czemu łatwiej można zrozumieć detale pełnego algorytmu AES [1].
\subsection{Arytmetyka ciał skończonych}
W Algorytmie AES wszystkie obliczenia wykonywane są na 8-bitowych bajtach, dodawanie,
mnożenie i dzielenie wykonywane są w ciele skończonym GF(28). Mówiąc skrótowo, ciało to
zbiór, w którym dodawanie, odejmowanie, mnożenie i dzielenie są działaniami wewnętrznymi:
wynik dowolnej z tych operacji wykonanej na elementach zbioru również należy do tego zbio-
ru. Dzielenie definiowane jest jako mnożenie przez element odwrotny: a/b oznacza to samo,
co a(b-1).\\

\noindent Wszystkie niemal algorytmy szyfrujące, zarówno te konwencjonalne, jak i te z kluczami pu-
blicznymi , obejmują obliczenia na liczbach całkowitych. Jeżeli jedną z operacji jest dzielenie,
wymagana jest arytmetyka zdefiniowana nad ciałem: wykonalność dzielenia wiąże się bowiem
z wymaganiem, by każdy nie zerowy element ciała posiadał odwrotność multiplikowaną. Ze
względu na wygodę obliczeń oraz efektywności implementacji chcielibyśmy operować liczbami
dającymi się zapisywać na ustalonej (n) liczbie bitów i jednocześnie w pełni wykorzystują-
cymi zakres wartości (od 0 do 2n1) oferowanych przez tę liczbę bitów. Niestety, zbiór Z2n ze
zwykłą arytmetyką modulo 2n nie jest ciałem, ponieważ żadna liczba parzysta nie posiada w
tej arytmetyce odwrotności multiplikowanej nie dla każdego elementu b istnieje takie
x, że bx mod2n=1.\\

\noindent Możliwe jest jednak takie zdefiniowanie działań w zbiorzeZ2n, by stał się ciałem GF(2n).
Rozpatrzymy zbiór S wielomianów stopnia nie wyższego niż n - 1 ze współczynnikami binar-
nymi (czyli pochodzącymi ze zbioru{0,1}). Każdy z tych wielomianów ma postać

$$f(x)=a_{n-1}x^{n-1}+a_{n-2}x^{n-2}+...+a_{1}x+a_{0}= \sum_{i=0}^{n-1} a_{i}x^{i} $$
gdzie każde ai ma wartość 0 albo 1. W zbiorze S istnieje wówczas dokładnie 2n różnych
wielomianów; dla n = 3 jest to 8 następujących wielomianów:
$$
\begin{array}{cccc}
0 & x & x^{2} & x^{2}+x \\
1 & x+1 & x^{2}+1 & x^{2}+x+1 \\
\end{array}
$$ 
\newpage
\noindent \textbf{Można sprawić, że zbiór S będzie ciałem skończonym, definiując odpowiednio operacje na
jego elementach:}
\newline
\begin{enumerate}
\item  Operacje te opierać się muszą generalnie na tradycyjnych regułach arytmetyki wielomianowej, jednak z zastrzeżeniami wymienionymi w punktach 2. i 3.
\item Operacje na współczynnikach wielomianów wykonywane są modulo 2; dodawanie jest
wówczas równoważne operacji XOR.
\item Gdy wynikiem mnożenia wielomianów jest wielomian f(x) stopnia wyższego niż n -
1, należy wielomian ten zredukować modulo pewien nieredukowalny wielomian m(x)
stopnia n - czyli zamiast oryginalnego wielomianu f(x) wziąć resztę r(x) z jego dzielenia
przez m(x), oznaczoną f(x) mod m(x). Wielomian nazywamy nieredukowalnym, jeżeli
nie można go przedstawić w postaci iloczynu dwóch wielomianów niższych stopni.
\end{enumerate}
\begin{table}[h]
\centering\caption{Parametry algorytmu AES[2]}\label{tabela 1}
\begin{tabular}{|c|c|c|c|}
\hline
Długość klucza(w słowach/bajtach/bitach) & 4/16/128 & 6/24/192 & 8/32/256 \\
\hline
Rozmiar bloku wejściowego (w słowach/bajtach/bitach) & 4/16/128 & 4/16/128 & 4/16/128 \\
\hline
Liczba rund & 10 & 12 & 14 \\
\hline
Długość podklucza rundy (w słowach/bajtach/bitach) & 4/16/128 & 4/16/128 & 4/16/128 \\
\hline
Rozmiar rozwiniętego podklucza (w słowach/bajtach) & 44/176 & 52/208 & 60/240\\
\hline
\end{tabular}
\end{table}
\newpage
\section{Ciasto z jabłkami i żurawiną}
Ciasto z jabłkami i żurawiną [3] to fantastyczna szarlotka dla fanów żurawiny. Połączenie
jabłek i świeżej żurawiny jest bezbłędne. Ciasta nie trzeba chłodzić i wstępnie podpiekać,
dlatego przygotowanie nie powinno nam zająć dużo czasu. Świeża żurawina jest obecnie w
sezonie więc korzystajmy z tego dobrodziejstwa, choć w przepisie bez problemu i żadnych
dodatkowych zmian można użyć żurawiny mrożonej. Polecam!\\
\newline
\textbf{Składniki na ciasto}
\newline
\begin{itemize}
\item 300 g mąki pszennej
\item 200 g masła
\item 1 łyżeczka proszku do pieczenia
\item 1 łyżeczka zmielonego cynamonu
\item 1 duże jajko
\item 1 żółtko, z dużego jajka
\item 110g jasnego miałkiego brązowego cukru
\end{itemize}
Wszystkie składniki na ciasto umieścić w misie malaksera i zmiksować do połączenia. Otrzy-
mane ciasto będzie bardzo gęste, wręcz „ciasteczkowo” gęste. Jeśli ciasto będzie mocno klejące
można je schłodzić w lodówce przez około 60 minut. \\

\noindent Kwadratową formę o boku 24 cm wyłożyć papierem do pieczenia, sam spód. Ciasto podzielić
na 2 równe części. Pierwszą częścią ciasta wylepić spód formy, na nie równomiernie wyło-
żyć nadzienie jabłkowo – żurawinowe, a na górę wyłożyć drugą część ciasta – w kawałeczkach. \\

\noindent Ciasto z jabłkami i żurawiną piec w temperaturze 180ºC, najlepiej bez termoobiegu (czyli
z grzaniem góra-dół), przez około 50 minut. Po wystudzeniu można oprószyć dodatkowym
cukrem pudrem. \\
\newline
\textbf{Składniki na nadzienie jabłkowo - żurawinowe}
\newline
\begin{itemize}
\item 600 g jabłek odmiany szarlotkowej
\item 300 g żurawiny
\item 1/4 łyżeczka cukru (lub mniej,do smaku)
\item 1 łyżeczka zmielonego cynamonu
\item 1 łyżeczka soku z cytryny
\item 1 łyżka skrobi ziemniaczanej
\end{itemize}
\newpage
\noindent Jabłka umyć, obrać, usunąć gniazda nasienne a następnie pokroić w półplasterki.\\

\noindent Pokrojone jabłka umieścić w garnku, dodać żurawinę, cukier, cynamon i sok z cytryny, na-
stępnie wymieszać. Pogotować przez kilka minut do wstępnego odparowania wody i popękana
żurawiny. Następnie oprószyć przez sitko skrobią ziemniaczaną i wymieszać. Zdjąć z palnika,
nie trzeba studzić (można gorące wyłożyć na przygotowane ciasto w formie).\\
\end{document}