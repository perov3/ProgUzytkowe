\documentclass[12pt, letterpaper, titlepage]{article}
\usepackage[left=3.5cm, right=2.5cm, top=2.5cm, bottom=2.5cm]{geometry}
\usepackage[MeX]{polski}
\usepackage[utf8]{inputenc}
\usepackage{graphicx}
\usepackage{enumerate}
\usepackage{amsmath} %pakiet matematyczny
\usepackage{amssymb} %pakiet dodatkowych symboli
\title{Pierwszy dokument LaTeX}
\author{Patryk Śmiłek}
\date{Październik 2022}
\begin{document}
\maketitle
\begin{enumerate}
\item punkt1
\item punkt2
\item PUNKT3
\item punkT4
\end{enumerate}
\newpage
\textbf{Przepis na babeczki}
\newline
\begin{enumerate}
\item mąki pszennej
\item 100g cukru
\item łyżeczki cukru waniliowego
\item jajka
\item 150ml gazowanej wody mineralnej
\item 100ml oleju
\item 2 łyżeczki proszku do pieczenia
\item 100g posiekanej czekolady (nie za drobno) lub groszku czekoladowego
\end{enumerate}
\begin{enumerate}
\item W jednej misce wymieszać mąkę, cukier, cukier waniliowy i proszek do pieczenia.
\item W drugiej misce wymieszać trzepaczką gazowaną wodę mineralną, olej i jajka.
\item Następnie wymieszać trzepaczką składniki z obu misek. Nie mieszać długo, lecz tylko do \item momentu połączenia się składników. Na końcu wmieszać delikatnie szpatułką kawałki czekolady.
\item Formę na muffinki wyłożyć papilotkami. Ciasto nałożyć do papilotek.
\item Muffinki piec w nagrzanym piekarniku ok. 30 minut, do suchego patyczka, w temperaturze 180°C (grzałka góra- dół). Studzić przez parę minut przy uchylonych drzwiczkach piekarnika. \item Następnie wyciągnąć na kratkę i pozostawić do całkowitego ostygnięcia.
\end{enumerate}
\end{document}